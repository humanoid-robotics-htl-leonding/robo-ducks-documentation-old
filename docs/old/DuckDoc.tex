\documentclass[12pt,a4paper]{article}
\usepackage{ducktex}
% Use latexmk for building, else it might throw strange errors
\author{    
    Jan Neuburger \\
    jan.neuburger.jn@gmail.com
    \and
    Erik Mayrhofer \\
    erik.mayrhofer@liwest.at
    \and
    Florian Schwarcz \\
    florian.schwarcz@gmx.at
    \and
    Maximilian Wahl \\
    mexx.wale@gmail.com
    }
\title{Duckburg\\[0.2em]Concept of a Framework}
\begin{document}
\maketitle
\newpage
\tableofcontents
\newpage
\section{The components}\label{sec:theComponents}
\subsection{The Brain}\label{subsec:theBrain}
The Brain is the central part of the Framework, even though being the simplest component upon the three.
It works as the junction for the communication traffic of the Engines.
The tasks of the Brain are only to provide Engines with the possibility to subscribe to Slots and send Intents to them.
Also it is the keeper of the dictionary that contains Slots and Filters.

The main focus of the brain is to minimize its logical effort and therefore maximize communication possibilities
for Engines.
\subsection{The Engines}\label{subsec:theEngines}
Engines are the components of Duckburg that do the thinking.
Basically an Engine can do two things:
\begin{itemize}
\item Send intents
\item Subscribe to Slots    
\end{itemize}
The key point of Engines is that they don't really "do" anything but that they "think" about and react to things.
\subsection{The Agents}\label{subsec:theAgents}
Agents are components of Duckburg that really make things go\dots\ literally.
They are invoked by their Engines and then perform the task they are made to do.
Examples for Agents could be: An Agent
\begin{itemize}
\item \dots\ that walks
\item \dots\ that calculates the position of an object
\item \dots\ that writes a log entry to a file
\item \dots\
\end{itemize}
An Agent shouldn't be a huge logical block but rather focus on executing a single task (Microservices). 
\section{Intent-Flow}\label{sec:intent-flow}
\subsection{Start}\label{subsec:start}
The brain is the first component that will start after the system boot.
It will then proceed to load a config file in which the filenames of all Engines are located.
Afterwards the Brain will load all Engines that are mentioned in the config file and initialize them.
\subsection{The Intent}\label{subsec:theIntent}
An Engine is able to send an Intent.
An Intent contains a command that should be executed.
After an Intent is sent to the Brain it runs through a number of Intent Filters that sort out illegal Intents.
Then the Brain forwards the Intent to the correct slot.
If the Slot that the Intent is sent to doesn't exist,
it is created.
An Intent can also be delayed, which means that after reaching the brain it will
not continue on to the dispatcher until the specified time has passed.
\subsection{Subscriptions}\label{subsec:subscriptions}
When an Intent reaches the Slot it is sent to all Engines that have subscribed to this Slot.
An Engine subscribes to a Slot by specifying a method that will be called when an Intent is sent to this Slot.
As soon as an Engine subscribes to a Slot this Slot is created if it didn't exist before.
\subsection{An example}\label{subsec:anExample}
The Perception Engine spotted the Ball in its last action.
It now sends an Intent "Move-To-Ball"
(in reality the Intent would consist of a certain formatted header but we keep it simple for the expample).
The Intent gets to the filters.
If the Filter for "We are penalized" would be active it wouldn't let this intent pass
because it is of type "Move" and we aren't allowed to move when penalized.
If the Intent passes the Filters it gets to the Brain, where the brain sends the Intent to the "Move-To-Ball" Slot.
\\
At this instant the Motion Engine is already subscribed to the "Move-To-Ball" Slot with its method "move".
From now on the "move" Method will be invoked when any Intent is sent to the "Move-To-Ball" Slot.
\\
Now the Intent from before gets to the Motion Engine which starts its "move" method.
In the Engines "move" it tells its Agents the right tasks to complete to move.
The Agents do what the Engine tells them to and the Robot moves towards the ball.
\\
It is possible to send new Intents when a method is invoked due to a subscription.

\section{Internal Structure}
\subsection{Intents}
An Intent has 3 Fields:
\begin{itemize}
    \item An id, that tells what slot the Intent should go to
    \item A field for the data that the Intent is carrying
    \item delay, that specifies what time after being isseued the Intent should
          be dispatched
\end{itemize} 
\end{document}