% https://en.wikibooks.org/wiki/LaTeX/Source_Code_Listings
\documentclass{article}
\usepackage{listings}
\usepackage{minted}
\usepackage{marginnote}
\usepackage{xparse}
\usepackage{enumitem}

\newlist{notes}{enumerate}{1}
\setlist[notes]{label=Note: ,leftmargin=*}

\newmint[bash]{bash}{breaklines}
\newcommand{\bbash}[1]{\bash|#1|}

\begin{document}
In this approach we try to compile the HULKs toolchain on a docker.
This is our \texttt{docker-compose.yml}

\begin{minted}[linenos]{yaml}
version: "3"

volumes:
  ubuntu-vol:

services:
  ubuntu:
    image: ubuntu:bionic
    volumes:
      - ubuntu-vol:/home
    entrypoint: /bin/bash
    stdin_open: true
    tty: true
\end{minted}
\paragraph{}
We log into our ubuntu with
\texttt{docker exec -it [containername] /bin/bash}.

\bash'apt-get update'
\bash'apt-get install git wget -y'

From the HULKs-Team-Research-Report: You need these Packages to 
build the toolchain.
\begin{quote}
build-essentials (gcc, make, ...), git, automake, autoconf, gperf, bison, flex, texinfo,
libtool, libtool-bin, gawk, libcursesX-dev, unzip, CMake, libexpat-dev, python2.7-dev,
nasm, help2man, ninja
\end{quote}

\bash'apt install build-essential -y'
\bash'apt install automake autoconf -y'
\bash'apt install gperf -y'
\bash'apt install bison flex texinfo -y'
\bash'apt install libtool libtool-bin gawk -y'

\paragraph{Note:}
We are not entirely sure if \texttt{libncurses5-dev} is the correct package
to satisfy \texttt{libcursesX-dev} but we think the \texttt{X} stands for
a random Versionnumber and the n was forgotten.
\bash'apt install libncurses5-dev -y'

\bash'apt install unzip cmake -y'
\paragraph{Note:} We are pretty sure, that \texttt{libexpat1-dev} is the correct
package for \texttt{libexpat-dev}.
\bash'apt install libexpat1-dev -y'

\bash'apt install python2.7-dev nasm help2man -y'

\paragraph{Note:} We think \texttt{ninja-build} is the correct package for \texttt{ninja}.
\bash'apt install ninja-build -y'

Our working directory will be \texttt{/home/docker} as \texttt{docker}
\begin{minted}{bash}
useradd docker
usermod -s /bin/bash docker
mkdir /home/docker
chown docker:docker /home/docker
cd /home/docker
\end{minted}

\bash'git clone https://github.com/Obyoxar/HULKsCodeRelease.git'

\bash'cd /home/docker/HULKsCodeRelease/tools/ctc-hulks'

\bash'./0-clean'

\paragraph{Note:} Before we can init the setup script,
we need to create a folder \texttt{ctc-hulks-config}.
\bash'mkdir ctc-hulks-config'
\bash'touch ctc-hulks-config/.config'
\bash'./1-setup'
\bash'./2-setup'

\end{document}
